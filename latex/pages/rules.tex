\subsubsection{Ανίχνευση Path Traversal}
\begin{lstlisting}
if (path_traversal_keywords in original_URI) {
    print fmt("[ALERT] Potential Path traversal detected. 
    Host: %s:%d, URI: %s", c$id$orig_h, c$id$orig_p, original_URI);
}
\end{lstlisting}
Ο πρώτος κανόνας εξετάζει αν το \textbf{URI} (Uniform Resource Identifier) που περιλαμβάνεται στο αίτημα περιέχει λέξεις-κλειδιά που σχετίζονται με τεχνικές path traversal, όπως \texttt{ ../, ..\\, /etc/passwd} ή άλλες παρόμοιες ακολουθίες. Αν ανιχνευτεί τέτοια λέξη-κλειδί, εμφανίζεται ειδοποίη-ση που περιλαμβάνει τη διεύθυνση IP του αιτούντος, την θύρα προέλευσης και το \textbf{URI} που θεωρείται ύποπτο.
\begin{lstlisting}
[ALERT] Potential Path traversal detected.
\end{lstlisting}
περιλαμβάνοντας τη διεύθυνση IP (\texttt{c\$id\$orig\_h}), την θύρα (\texttt{c\$id\$orig\_p}), και το \textbf{URI}.

Ο σκοπός αυτού του κανόνα είναι να εντοπίζει προσπάθειες παράκαμψης του προβλεπόμενου εύρους πρόσβασης σε αρχεία ή καταλόγους του διακομιστή. Αυτές οι επιθέσεις χρησιμοποιούνται συνήθως για την απόκτηση πρόσβασης σε ευαίσθητα δεδομένα ή αρχεία διαμόρφωσης που δεν θα έπρεπε να είναι προσβάσιμα από εξωτερικούς χρήστες.

\subsubsection{Ανίχνευση Code Injection}
\begin{lstlisting}
if (code_injection_keywords in full_body) {
    print fmt("[ALERT] Potential Code Injection detected. 
    Host: %s:%d, Body: %s", c$id$orig_h, c$id$orig_p, full_body);
}
\end{lstlisting}
Ο δεύτερος κανόνας ελέγχει το σώμα του αιτήματος (\textbf{full body}) για λέξεις-κλειδιά που υποδηλώνουν πιθανή \textbf{έγχυση κώδικα} (\textbf{Code Injection}), όπως \texttt{eval(), system(), ;} ή άλλες κακόβουλες εντολές. Αν τέτοια λέξη βρεθεί, εμφανίζεται ειδοποίηση που περιλαμβάνει τη διεύθυνση IP, την θύρα, και το πλήρες σώμα του αιτήματος.
\begin{lstlisting}
[ALERT] Potential Code Injection detected.
\end{lstlisting}
περιλαμβάνοντας τη διεύθυνση IP (\texttt{c\$id\$orig\_h}), την θύρα (\texttt{c\$id\$orig\_p}), και το πλήρες σώμα του αιτήματος (\texttt{full\_body}).
Ο σκοπός αυτού του κανόνα είναι η αποτροπή επιθέσεων στις οποίες οι εισβολείς προσπαθούν να εκτελέσουν αυθαίρετο κώδικα στον διακομιστή, όπως εκτέλεση \textbf{Shell Commands}, \textbf{SQL Injection}, ή άλλες παρόμοιες επιθέσεις. Η ανίχνευση τέτοιων μοτίβων μπορεί να βοηθήσει στην πρόληψη ζημιών όπως η κλοπή δεδομένων ή η αλλοίωση λειτουργιών του συστήματος.

\subsubsection{Έλεγχος μεγάλου Payload (buffer overflow)}
\begin{lstlisting}
if (|full_body| > payload_threshold) {
    print fmt("[ALERT] Payload exceeds threshold. 
    Host: %s:%d, Length: %s", c$id$orig_h, c$id$orig_p, |full_body|);
}
\end{lstlisting}
Ο τρίτος κανόνας ελέγχει το μέγεθος του σώματος του αιτήματος (payload) και το συγκρίνει με ένα \textbf{προκαθορισμένο όριο} (\textbf{payload threshold}). Αν το μέγεθος υπερβαίνει αυτό το όριο, ενεργοποιείται ειδοποίηση που περιλαμβάνει τη διεύθυνση IP, την θύρα, και το μήκος του σώματος.
\begin{lstlisting}
[ALERT] Payload exceeds threshold.
\end{lstlisting}
περιλαμβάνοντας τη διεύθυνση IP, την θύρα, και το μέγεθος του σώματος.
Ο σκοπός αυτού του κανόνα είναι να εντοπίσει περιπτώσεις όπου οι εισβολείς στέλνουν υπερβολικά μεγάλα δεδομένα σε μια προσπάθεια να προκαλέσουν ζημιά, όπως μέσω επιθέσεων \textbf{Βuffer Οverflow} ή κατάχρησης πόρων του διακομιστή.
