Στο πλαίσιο της υλοποίησης του Intrusion Detection System (IDS) με το εργαλείο \textbf{Zeek}, δοκιμάστη-καν διάφορες τεχνικές επιθέσεων για την αξιολόγηση της αποτελεσματικότητάς του. Οι επιθέσεις χωρίζονται σε δύο κατηγορίες: χειροκίνητες επιθέσεις και αυτοματοποιημένες επιθέσεις, οι οποίες αποσκοπούν στον εντοπισμό και την ανίχνευση κακόβουλων ενεργειών από το IDS.

Αρχικά, πραγματοποιήθηκαν χειροκίνητες επιθέσεις με στόχο την εκμετάλλευση αδυναμιών σε εφαρμογές και συστήματα. Μια από τις πιο συνηθισμένες επιθέσεις ήταν το \textbf{Cross-Site Scripting} (\textbf{XSS}), το οποίο περιλαμβάνει την εισαγωγή κακόβουλου \textbf{JavaScript} κώδικα σε ιστοσελίδες, με σκοπό την εκτέλεσή του στον browser του χρήστη. Στην προκειμένη περίπτωση, χρησιμοποιήθηκε ο παρακά-τω κώδικας για να προκαλέσει την εμφάνιση του περιεχομένου του cookie του χρήστη:
\begin{lstlisting}
<img src="" onerror="alert(document.cookie);" />
\end{lstlisting}

Ένα ακόμα παράδειγμα επίθεσης ήταν η \textbf{SQL Injection}, η οποία εκμεταλλεύεται ευπάθειες στις \textbf{SQL} ερωτήσεις μιας εφαρμογής για την εκτέλεση κακόβουλων εντολών στη βάση δεδομένων. Για παράδειγμα, χρησιμοποιήθηκε η εξής εντολή:
\begin{lstlisting}
INSERT INTO Posts (title, content) VALUES ('test1', 'test2'); DROP TABLE Posts; --
\end{lstlisting}

Η επίθεση αυτή αποσκοπούσε στην εκτέλεση ενός κακόβουλου \textbf{SQL statement} για τη διαγραφή δεδομένων από τη βάση. Παράλληλα, πραγματοποιήθηκε και επίθεση \textbf{Path Traversal}, η οποία στόχευε στην πρόσβαση σε ευαίσθητα αρχεία του συστήματος μέσω κακόβουλων αιτημάτων, όπως φαίνεται στο παρακάτω παράδειγμα:
\begin{lstlisting}
curl http://localhost/api/etc/passwd
\end{lstlisting}

Αυτού του είδους οι επιθέσεις χρησιμοποιούνται για την υποκλοπή κρίσιμων συστημικών αρχείων.

Επιπλέον, εξετάστηκε η επίθεση \textbf{Buffer Overflow}, στην οποία αποστέλλονται δεδομένα μεγαλύτε-ρα του αναμενόμενου μεγέθους για να προκαλέσουν υπερχείλιση μνήμης. Η συγκεκριμένη επίθεση εκτελέστηκε με την αποστολή μεγάλου αριθμού χαρακτήρων μέσω \textbf{HTTP αιτήματος}:
\begin{lstlisting}
curl -H "Content-Type: application/json" -d '{"title":"hello world", "content":"'$
(python -c 'print("A"*5050)' | sed 's/"/\\"/g')'"}' -X POST http://localhost:1234
\end{lstlisting}

Τέλος, πραγματοποιήθηκε και μια επίθεση \textbf{Distributed Denial of Service} (\textbf{DDoS}), η οποία είχε ως στόχο τη φόρτωση του συστήματος με μεγάλο όγκο αιτημάτων, μέσω του \textbf{script ./ddos.sh}.

Από την άλλη πλευρά, χρησιμοποιήθηκαν και αυτοματοποιημένα εργαλεία για την εκτέλεση επιθέ-σεων μεγάλης κλίμακας. Ένα από τα πρώτα εργαλεία ήταν το \textbf{hping3}, το οποίο χρησιμοποιεί \textbf{TCP SYN} πακέτα για την εκτέλεση \textbf{DDoS} επιθέσεων. Ένα παράδειγμα της εντολής που χρησιμοποιήθηκε είναι η εξής:
\begin{lstlisting}
sudo hping3 -i u40 -S -p 1234 -c 1000000 172.21.0.3
\end{lstlisting}

Ένα άλλο εργαλείο που χρησιμοποιήθηκε ήταν το \textbf{Apache Bench} (\textbf{ab}), το οποίο χρησιμοποιείται για την αποστολή μεγάλου αριθμού \textbf{HTTP} αιτημάτων σε έναν server, προκειμένου να εξεταστεί η αντοχή του συστήματος υπό φορτίο. Η εντολή για την εκτέλεση της επίθεσης είναι:
\begin{lstlisting}
ab -n 100000 -c 100 http://localhost:1234/
\end{lstlisting}

Ένα ακόμα εργαλείο ήταν το \textbf{Oha}, το οποίο χρησιμοποιεί παρόμοια λογική με το \textbf{Apache Bench}, αλλά είναι σχεδιασμένο για εξαιρετικά υψηλές επιδόσεις. Η εντολή που χρησιμοποιήθηκε για την εκτέλεση του εργαλείου είναι:
\begin{lstlisting}
oha -z 2m -c 1000 http://localhost:1234
\end{lstlisting}

Τέλος, χρησιμοποιήθηκε το εργαλείο \textbf{Nmap} για την ανίχνευση ανοιχτών θυρών και την εκτίμηση της ασφάλειας του συστήματος μέσω σάρωσης υπηρεσιών και συσκευών.

Οι παραπάνω επιθέσεις καλύπτουν ένα ευρύ φάσμα κακόβουλων ενεργειών, οι οποίες αναμένεται να ανιχνευθούν και να καταγραφούν από το IDS. Συμπερασματικά, είναι φανερό ότι το \textbf{Zeek} ήταν αποτελεσματικό στην ανίχνευση των περισσότερων από αυτές τις επιθέσεις. Ωστόσο, παρατηρήθηκαν περιοχές που απαιτούν περαιτέρω βελτιώσεις, όπως η παραμετροποίηση των κανόνων για την καλύτε-ρη αναγνώριση επιθέσεων με υψηλή συχνότητα, όπως οι \textbf{DDoS} επιθέσεις.
