Η συγκεκριμένη εξαμηνιαία εργασία επικεντρώνεται στην ανάπτυξη Συστήματος Ανίχνευσης Επίθε-σης (Intrusion Detection System - IDS) με χρήση του ανοιχτού κώδικα λογισμικού ανάλυσης δικτυακής κίνησης, \textbf{Zeek}. 

Η εφαρμογή του συστήματος γίνεται σε τοπικά αποθηκευμένη ιστοσελίδα τύπου ιστολογίου (blog), η οποία αποτελείται από 3 διαδραστικά μέρη (τίτλος, κείμενο, δημοσίευση) όπου ο χρήστης μπορεί να επέμβει μόνο στον τίτλο και το κείμενο. Οι πληροφορίες αποθηκεύονται σε βάση δεδομένων και εμφανίζονται στο κάτω μέρος της ιστοσελίδας. 

Επιγραμματικά, για τον έλεγχο λειτουργικότητας της υλοποίησης, πραγματοποιούνται 5 χειροκίνη-τες επιθέσεις (\textbf{XSS, SQL Injection, NGINX Path Traversal, DDoS}) και αξιοποιούνται 3 αυτόματα εργα-λεία (\textbf{NMap, hping3, oha}) αντίστοιχα για τη διενέργεια επιθέσεων. Σκοπός της εργασίας είναι αξιοποίηση του \textbf{Zeek} για τον έγκαιρο εντοπισμό τους με χρήση κανόνων εντός αυτού. 